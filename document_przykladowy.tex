
\documentclass{article}
\usepackage[T1]{fontenc}
\usepackage{graphicx}
\usepackage{amsmath, amsthm, amssymb}
\usepackage{hyperref}
\usepackage{polski}
\newtheorem{theorem}{Twierdzenie}
\theoremstyle{definition}
\usepackage{verbatim}
\newtheorem{definition}{Definicja}
\usepackage{listings}
\usepackage{color}
\usepackage{enumerate}
\title{Tytuł}
\author{Amelia Dorożko}
\date{\today}
\begin{document}
	
	\maketitle
	\begin{table}
		\centering
		\begin{tabular}{|c|c|c|}
			\hline
			Kolumna 1 & Kolumna 2 & Kolumna 3 \\ \hline
			Wiersz 1  & Dane 1    & Dane 2    \\ \hline
			Wiersz 2  & Dane 3    & Dane 4    \\ \hline
		\end{tabular}
		\caption{Przykładowa tabela}
	
	\end{table}
	\section{Sekcja numerowana}
	Przykładowy tekst.
	\subsection{Podsekcja numerowana}
	Przykładowy tekst.
	\section*{Sekcja nienumerowana}
	Przykładowy tekst.
	\subsection*{Podsekcja nienumerowana}
	Przykładowy tekst.
	\section{Środowiska matematyczne}
	\begin{definition}
		Przykładowa definicja.
	\end{definition}
	\begin{theorem}
		Przykładowe twierdzenie.
	\end{theorem}
	

	
	\subsection*{Wyliczenia}
	
	Przykładowo obliczymy:
	
	\[
	\int_{-1}^{1} x^3 \, dx
	\]
	Korzystamy z podstawowego wzoru na całkowanie funkcji potęgowej:
	
	\[
	\int x^n \, dx = \frac{x^{n+1}}{n+1} + C 
	\] \\Stosując ten wzór do \( x^3 \), otrzymujemy:
	
	\[
	\int x^3 \, dx = \frac{x^4}{4} + C
	\]\\Teraz obliczamy wartość całki oznaczonej:
	
	\[
	\int_{-1}^{1} x^3 \, dx = \left[ \frac{x^4}{4} \right]_{-1}^{1}
	\]\\Podstawiamy granice całkowania:
	
	\[
	\frac{1^4}{4} - \frac{(-1)^4}{4} = \frac{1}{4} - \frac{1}{4} = 0
	\]\\Ostateczny wynik to:
	
	\[
	\int_{-1}^{1} x^3 \, dx = 0
	\]
	
	
	
	\begin{theorem}{(O wartości średniej)}
		Niech $g$, $g'$, $f$, są ciągłe na $[b_1, b_2]$, $g$ monotoniczne na $[b_1, b_2]$. Wtedy $ \exists c \in [b_1, b_2]$ takie, że
		\begin{equation}\label{wartość średnhia całki }
			\int_{b_1}^{b_2} f(x)g(x) \, dx = g(b_1)\int_{b_1}^{c}f(x) \, dx + g(b_2)\int_{c}^{b_2}f(x)
		\end{equation}
	\end{theorem}
	
	\begin{theorem}{(Kryterium Dirichleta zbieżności całek)}
		Niech:
		
		1$^{\circ}$ $\exists M > 0 \ \forall c \in (a,\infty), \ \ f \in \mathbb{R}[a,c]$ oraz $\left|\int_{a}^{c} f(x) \, dx\right| \leq M$.
		
		2$^{\circ}$ $g(x)$ jest monotoniczna na $[a,\infty)$.
		
		3$^{\circ}$ $\lim_{x \to \infty} g(x) = 0$.
		
		Wtedy
		
		\[
		\int_{a}^{\infty} f(x) g(x) \, dx \ \text{jest zbieżna}.
		\]
	\end{theorem}
	\begin{proof}
		(z dowodu przez różniczkowanie złożenia: $g, f \in C^1$ i są ograniczone na $[a, \infty)$)
		
		\[
		\text{Ustalmy } \varepsilon > 0
		\]
		
		\[
		\exists b_0 > a \quad \forall x > b_0 \quad 0 < g(x) < \frac{\varepsilon}{8M}
		\]
		
		Wybieramy dowolne $b_1, b_2$ spełniające $b_0 < b_1 < b_2 < \infty$.
		Z II tw. o wartości średniej:
		\[
		\exists c \in [b_1,b_2] \text{ takie, że } \int_{b_1}^{b_2} f(x)g(x) \, dx = g(b_1)\int_{b_1}^{c}f(x) \, dx + g(b_2)\int_{c}^{b_2}f(x)
		\]
		Stąd
		\[
		\left|\int_{b_1}^{b_2} f(x) g(x) \, dx\right| \leq g(b_1) \left| \int_{b_1}^{c} f(x) \, dx \right| + g(b_2) \left|\int_{c}^{b_2} f(x) \, dx \right| \leq \frac{\varepsilon}{2M}
		\]
		\[
		\left|\int_{b_1}^{c} f(x) \, dx\right| = \left|\int_{a}^{c} f(x) \, dx - \int_{a}^{b_1} f(x) \, dx \right| \leq \left|\int_{a}^{c} f(x) \, dx \right| + \left| \int_{a}^{b_1} f(x) \, dx \right| \leq 2M
		\]
		Podobnie
		\[
		\left|\int_{c}^{b_2} f(x) \, dx \right|= \left| \int_{a}^{b_2} f(x) \, dx - \int_{a}^{c} f(x) \, dx \right| \leq \left| \int_{a}^{b_2} f(x) \, dx \right| +  \left| \int_{a}^{c} f(x) \, dx \right| \leq 2M
		\]
		spełnia warunek Cauchy'ego zbieżności całki.
	\end{proof}
		
	\subsubsection*{Przykład}
	Udowodnij, że podana całka:
	\[
	\int_{1}^{\infty} \frac{\sin x}{x} \, dx,
	\] jest zbieżna.\\ Korzystając z kryterium Dirichleta:
	\[
	f(x) = \sin x, \quad g(x) = \frac{1}{x}
	\]
	\[
	1^\circ \quad \forall \, c \in (1, \infty) \left|\int_{1}^{c} \sin x \, dx\right| = \left|\left[\cos x\right]_{1}^{c}\right| = \left|\cos c - \cos 1\right| \leq \left|\cos c\right| + \left|\cos 1\right| \leq 2
	\]
	\[
	2^\circ \quad g(x) \text{ jest monotoniczna na } [1, \infty)
	\]
	\[
	3^\circ \quad \lim_{x \to \infty} \frac{1}{x} = 0
	\]
	
	\text{Z kryt. Dirichleta}
	\[
	\int_1^{\infty} \frac{\sin x}{x} \, dx \text{ jest zbieżna.}
	\]


\subsubsection*{Przykładowy plik pdf}
\begin{figure}
	\centering
	\includegraphics[width=0.9\textwidth]{wykres(4)1.pdf}
	
\end{figure}
\subsubsection*{Pakiet verbatim}
\begin{verbatim}
	Przykładowy tekst. 
\end{verbatim}

\subsubsection*{Wypunktowanie}
\begin{enumerate}[a)]
	\item wypunktowanie,
	\item wypunktowanie.
\end{enumerate}

\begin{enumerate}
	\item wypunktowanie,
	\item wypunktowanie.
\end{enumerate}
\begin{itemize}
	\item wypunktowanie,
	\item wypunktowanie.
\end{itemize}



\subsubsection*{Fragment kodu}

\lstset{
	language=Python,        
	basicstyle=\ttfamily,    
	keywordstyle=\color{blue},
	commentstyle=\color{green},
	stringstyle=\color{red},    
	showstringspaces=false,
	numbers=left,           
	tabsize=4               
}


	
	\begin{lstlisting}
		#Przykladowy kod w Pythonie
		def dodawanie(x,y):
			x+y = z
			return z
		print("Wynik dodawania:" + dodawnaie(5,6))
		
	\end{lstlisting}

\begin{thebibliography}{9}
	
	\bibitem{latex} 
	Autor, 
	\textit{Tytuł}, 
	Rok powstania.
	\bibitem{latex}
	Autor2, 
	\textit{Tytuł2},
	Rok powstania

	
\end{thebibliography}
	
	
	
	
	
	
	
\end{document}